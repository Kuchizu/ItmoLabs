\documentclass{article}
\usepackage[utf8]{inputenc}
\usepackage{amsmath, amsfonts, amssymb}
\usepackage[russian]{babel}
\usepackage[a4paper, total={170mm,257mm},left=10mm, top=10mm, right=20]{geometry}
\usepackage{tikz}

\begin{document}

\textbf{Задание 5. Приложения определенного интеграла} \\

\textbf{Условие задачи:} \\
Определить массу круглого конуса высотой $4$ м и диаметром основания $6$ м, если
плотность конуса в каждой точке равна квадрату расстояния этой точки от плоскости,
проходящей через вершину конуса параллельно его основанию. \\

\begin{center}
\begin{tikzpicture}[scale=0.8]
    % Основание конуса
    \draw (0,0) ellipse (3cm and 1.2cm);
    
    % Вершина конуса
    \draw (0,4) -- (-3,0);
    \draw (0,4) -- (3,0);
    
    % Высота и радиус конуса
    \draw[-{Latex[length=2mm]}] (0,0) -- (0,4) node[midway,left] {$4\,\text{м}$};
    \draw[-{Latex[length=2mm]}] (0,0) -- (3,0) node[midway,below] {$3\,\text{м}$};
    
    % Боковые линии
    \draw[dashed] (0,0) -- (0,4);
    \draw[dashed] (0,0) -- (3,0);

    % Плоскость параллельно основанию
    \draw[dashed] (-3,2.5) -- (3,2.5);
    \draw[-{Latex[length=2mm]}] (0,0) -- (0,2.5) node[midway,left] {$x$};

\end{tikzpicture}
\end{center}

\\

\textbf{Математическая модель задачи:} \\
Пусть $V$ - объем элементарного слоя конуса толщиной $dx$, который удален от вершины конуса на расстоянии $x$. Тогда масса $dm$ этого элементарного слоя равна произведению плотности $\rho(x)$ на объем $V$.

$$
dm = \rho(x) V
$$

\textbf{Разбиение промежутка изменения аргумента на элементарные участки $dx$:} \\
Высота конуса равна $4$ м. Будем разбивать высоту конуса на элементарные участки $dx$.

\textbf{Вычисление малого приращения искомой величины $Q$ на элементарном участке $dx$:} \\
Из условия задачи плотность $\rho(x) = x^2$. Объем элементарного слоя конуса $V$ равен произведению площади основания $S$ на толщину слоя $dx$. 

Поскольку радиус конуса изменяется линейно от $0$ до $3$ м, можно использовать пропорцию для вычисления радиуса $r(x)$ на расстоянии $x$ от вершины:

$$
r(x) = \frac{3x}{4}
$$

Теперь вычислим площадь основания $S$ для элементарного слоя:

$$
S = \pi r^2(x) = \pi \left(\frac{3x}{4}\right)^2 = \frac{9\pi x^2}{16}
$$

Теперь можем вычислить массу элементарного слоя $dm$:

$$
dm = \rho(x) V = x^2 \cdot \frac{9\pi x^2}{16} dx = \frac{9\pi x^4}{16} dx
$$

\textbf{Получение интегральной суммы:} \\
Чтобы вычислить общую массу конуса $M$, нужно просуммировать массы всех элементарных слоев:

$$
M = \int dm = \int_{0}^{4} \frac{9\pi x^4}{16} dx
$$

\textbf{Вычисление определенного интеграла с использованием первообразной и формулы Ньютона-Лейбница:} \\
Найдем первообразную подынтегральной функции:

$$
\int \frac{9\pi x^4}{16} dx = \frac{9\pi}{16} \int x^4 dx = \frac{9\pi}{16} \cdot \frac{x^5}{5} + C
$$

Теперь применим формулу Ньютона-Лейбница для вычисления определенного интеграла:

$$
M = \int_{0}^{4} \frac{9\pi x^4}{16} dx = \left[\frac{9\pi}{16} \cdot \frac{x^5}{5}\right]_0^4 = \frac{9\pi}{16} \cdot \frac{4^5}{5} - \frac{9\pi}{16} \cdot \frac{0^5}{5}
$$

Упрощаем:

$$
M = \frac{9\pi}{16} \cdot \frac{1024}{5} = \frac{9216\pi}{80}
$$

\textbf{Ответ:} \\
Масса круглого конуса равна $\frac{9216\pi}{80}$ кг.

\end{document}

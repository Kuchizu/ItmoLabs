\documentclass[12pt]{article}
\usepackage{amsmath, amssymb, amsthm}
\usepackage{graphicx}
\usepackage{pgfplots}
\pgfplotsset{compat=1.16}
\usepackage[russian]{babel}
\usepackage[a4paper, total={170mm,257mm},left=9mm, top=9mm, right=30]{geometry}

\begin{document}

\section{Введение}
В данном отчете рассмотрена задача о линейном операторе $A$, действующем в трехмерном вещественном пространстве $R^3$, который проецирует пространство на подпространство $L1$ параллельно подпространству $L2$. Подпространства $L1$ и $L2$ заданы уравнениями $x = 0$ и $2x = 2y = -z$ соответственно.

\section{Описание подпространств L1 и L2}

Подпространство $L1$ определено уравнением $x = 0$. Это означает, что все точки в этом подпространстве имеют координаты вида $(0, y, z)$, где $y$ и $z$ - произвольные вещественные числа. Подпространство $L2$ определено уравнениями $2x = 2y = -z$. Деление обеих частей уравнения на 2 дает $x = y = -\frac{z}{2}$. Это означает, что все точки в этом подпространстве имеют координаты вида $(t, t, -2t)$, где $t$ - произвольное вещественное число.

\begin{figure}[h]
\centering
\begin{tikzpicture}
\begin{axis}[view={120}{30}]
\addplot3[surf, domain=-3:3, domain y=-3:3, opacity=0.5] {0};
\addplot3[surf, domain=-3:3, domain y=-3:3, opacity=0.5] {-2*x};
\end{axis}
\end{tikzpicture}
\caption{Подпространства $L1$ (плоскость $x=0$) и $L2$ (плоскость $x=y=-\frac{z}{2}$)}
\end{figure}

\section{Определение линейного оператора A}
Линейный оператор $A$ определен как оператор проекции пространства $R^3$ на подпространство $L1$ параллельно подпространству $L2$. Он может быть выражен следующим образом:

\begin{equation}
A(\vec{v}) = \vec{v} - \vec{proj}_{L2}\vec{v}
\end{equation}

Здесь $\vec{v}$ - произвольный вектор из $R^3$, а $\vec{proj}_{L2}\vec{v}$ - проекция $\vec{v}$ на $L2$. Проекция вектора на подпространство вычисляется по формуле:

\begin{equation}
\vec{proj}_{L2}\vec{v} = \frac{\vec{v} \cdot \vec{u}}{\vec{u} \cdot \vec{u}} \vec{u}
\end{equation}

где $\vec{u}$ - любой ненулевой вектор из $L2$. В качестве $\vec{u}$ мы можем выбрать вектор $(1, 1, -2)$.

Тогда формула для $A$ принимает следующий вид:

\begin{equation}
A(\vec{v}) = \vec{v} - \frac{\vec{v} \cdot (1, 1, -2)}{(1, 1, -2) \cdot (1, 1, -2)} (1, 1, -2)
\end{equation}

\section{Матрица линейного оператора A}
Матрица линейного оператора A в стандартном базисе $R^3$ ($\vec{i} = (1, 0, 0)$, $\vec{j} = (0, 1, 0)$, $\vec{k} = (0, 0, 1)$) может быть получена путем применения оператора к каждому из базисных векторов и записи результатов в столбцы матрицы. 

\begin{align*}
A(\vec{i}) &= (1, 0, 0) - \frac{(1, 0, 0) \cdot (1, 1, -2)}{(1, 1, -2) \cdot (1, 1, -2)} (1, 1, -2) = (1, 0, 0), \\
A(\vec{j}) &= (0, 1, 0) - \frac{(0, 1, 0) \cdot (1, 1, -2)}{(1, 1, -2) \cdot (1, 1, -2)} (1, 1, -2) = (0, 1, -1), \\
A(\vec{k}) &= (0, 0, 1) - \frac{(0, 0, 1) \cdot (1, 1, -2)}{(1, 1, -2) \cdot (1, 1, -2)} (1, 1, -2) = (0, -1, 1).
\end{align*}

Поэтому, матрица оператора A:

\begin{equation}
A = \begin{pmatrix} 1 & 0 & 0 \\ 0 & 1 & -1 \\ 0 & -1 & 1 \end{pmatrix}
\end{equation}

\section{Спектральный анализ и диагонализация матрицы}
Спектральный анализ используется для диагонализации матрицы оператора A. Для диагонализации матрицы, сначала нужно найти ее собственные значения и собственные векторы. Собственные значения матрицы A — это корни характеристического полинома $\det(A - \lambda I) = 0$. 

\begin{align*}
\det(A - \lambda I) &= \det\begin{pmatrix} 1-\lambda & 0 & 0 \\ 0 & 1-\lambda & 1 \\ 0 & 1 & 1-\lambda \end{pmatrix} \\
&= (1-\lambda) \det\begin{pmatrix} 1-\lambda & 1 \\ 1 & 1-\lambda \end{pmatrix} \\
&= (1-\lambda)^3 - 2(1-\lambda) \\
&= \lambda^3 - 2\lambda^2 + 2\lambda,
\end{align*}

следовательно, корни этого полинома являются собственными значениями матрицы A.

Собственные векторы находятся из уравнения $(A - \lambda I)\vec{v} = 0$ для каждого собственного значения $\lambda$.

\newpage
\section{Базис, в котором матрица линейного оператора A имеет диагональный вид}

\begin{minipage}{0.45\textwidth}
Для каждого собственного значения мы найдем собственный вектор и составим из этих векторов базис. В этом базисе матрица оператора A будет иметь диагональный вид с собственными значениями на диагонали.

\begin{align*}
\text{Для } \lambda_1 : &(A - \lambda_1 I)\vec{v} = 0, \\
\text{Для } \lambda_2 : &(A - \lambda_2 I)\vec{v} = 0, \\
\text{Для } \lambda_3 : &(A - \lambda_3 I)\vec{v} = 0.
\end{align*}

Для $\lambda_1 = 1$ мы имеем следующую систему уравнений:

\begin{align*}
0x + 0y + 0z &= 0, \\
0x + 0y + z &= 0, \\
0x + y + z &= 0.
\end{align*}

Эта система имеет решение $v_1 = (1, 0, 0)$.
\end{minipage}
\hfill
\begin{minipage}{0.45\textwidth}
Аналогично, для $\lambda_2 = 2$ и $\lambda_3 = 0$ получаем системы:

\begin{align*}
x + 0y + 0z &= 0, \\
0x - y + z &= 0, \\
0x - y + z &= 0,
\end{align*}

и 

\begin{align*}
-x + 0y + 0z &= 0, \\
0x - 1y + 1z &= 0, \\
0x - 1y + 1z &= 0.
\end{align*}

Они имеют решения $v_2 = (0, 1, -1)$ и $v_3 = (0, 1, 1)$ соответственно.

Таким образом, собственные векторы матрицы оператора $A$ равны:

\begin{align*}
v_1 &= (1, 0, 0), \\
v_2 &= (0, 1, -1), \\
v_3 &= (0, 1, 1).
\end{align*}
\end{minipage}

\vspace{1cm}

Базис в котором матрица оператора A имеет диагональный вид — это базис из собственных векторов. Базис можно использовать для перехода к другому представлению пространства, где оператор действует более простым образом.

\begin{figure}[h]
\centering
\begin{tikzpicture}
\begin{axis}[view={120}{30}]
\addplot3[surf, domain=-3:3, domain y=-3:3, opacity=0.5] {0};
\addplot3[surf, domain=-3:3, domain y=-3:3, opacity=0.5] {-2*x};
\addplot3[-stealth, red, thick] coordinates {(0,0,0) (1,0,0)};
\addplot3[-stealth, red, thick] coordinates {(0,0,0) (0,1,-1)};
\addplot3[-stealth, red, thick] coordinates {(0,0,0) (0,1,1)};
\end{axis}
\end{tikzpicture}
\caption{Подпространства $L1$ и $L2$ с базисом, состоящим из собственных векторов матрицы $A$}
\end{figure}


\end{document}

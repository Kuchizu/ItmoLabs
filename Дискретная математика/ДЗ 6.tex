\documentclass{article}

\usepackage[T2A]{fontenc}
\usepackage[utf8]{inputenc}
\usepackage{mathtools}
\usepackage{array}
\usepackage{multirow}
\usepackage[russian]{babel}
\usepackage{titling}
\usepackage[a4paper,left=2cm,right=2cm,top=0.5cm,bottom=2cm]{geometry}

\newcommand*{\carry}[1][1]{\overset{#1}}
\newcolumntype{B}[1]{r*{#1}{@{\,}r}}

\usepackage{enumitem}
\makeatletter
\AddEnumerateCounter{\asbuk}{\russian@alph}{щ}
\makeatother

\begin{document}
\begin{large}
\begin{center}
  \LARGE Дискретная математика\\
  Домашнее задание №6\\
  «Сложение чисел с плавающей запятой»\\
  Вариант №59\\
  Выполнил: Нодири Хисравхон (гр. P3131)\\
  СЛОЖЕНИЕ ЧИСЕЛ С ПЛАВАЮЩЕЙ ЗАПЯТОЙ\\
\end{center}
\end{large}
\begin{minipage}[t]{0.5\textwidth - 0.25cm - 3pt}
\setlength{\parskip}{0.5em}
\subsection*{Формат Ф1 (с 12 битной мантиссой)}
$A = 62.84 = 0.\text{3ED7}..._{16} \approx 0.\underbracket{\text{3ED}}_{\mathclap{\text{мантисса}}}\vphantom{0}_{\scriptstyle 16} * 16^{2}$ \\ $X_A = 2 + 64 = \underbracket{66}_{\mathclap{\text{характеристика}}}$

\begin{tabular}{ccc} \hline \multicolumn{1}{|c}{{0}} & \multicolumn{1}{|c|}{1\:0\:0\:0\:0\:1\:0} & \multicolumn{1}{|c|}{0\:0\:1\:1\:1\:1\:1\:0\:1\:1\:0\:1} \\ \hline \scriptsize 0 & \scriptsize 1 \hfill \scriptsize 7 & \scriptsize 8 \hfill \scriptsize 19 \end{tabular}

$B = 58.92 = 0.\text{3AEB}..._{16} \approx 0.\underbracket{\text{3AF}}_{\mathclap{\text{мантисса}}}\vphantom{0}_{\scriptstyle 16} * 16^{2}$ \\ $X_B = 2 + 64 = \underbracket{66}_{\mathclap{\text{характеристика}}}$

\begin{tabular}{ccc} \hline \multicolumn{1}{|c}{{0}} & \multicolumn{1}{|c|}{1\:0\:0\:0\:0\:1\:0} & \multicolumn{1}{|c|}{0\:0\:1\:1\:1\:0\:1\:0\:1\:1\:1\:1} \\ \hline \scriptsize 0 & \scriptsize 1 \hfill \scriptsize 7 & \scriptsize 8 \hfill \scriptsize 19 \end{tabular}

\textbf{Сравнение порядков:} \\
\setlength{\tabcolsep}{2pt}
\begin{tabular}{rcccccccl}
    \scriptsize \phantom{0}&\scriptsize \phantom{0}&\scriptsize \phantom{0}&\scriptsize \phantom{0}&\scriptsize \phantom{0}&\scriptsize \phantom{0}&\scriptsize \phantom{0} \\
    &1&0&0&0&0&1&0 & $= X_A$ \\
$-$ &1&0&0&0&0&1&0 & $= X_B$ \\ \hline
    &0&0&0&0&0&0&0 & $= (X_A - X_B)_\text{доп}$
\end{tabular}

$X_A - X_B = 0,\: X_C = X_A = 66$


\textbf{Оба операнда положительные:}


\setlength{\tabcolsep}{2pt}
\begin{tabular}{rcccccccccccccl}
    \scriptsize \phantom{0}&\scriptsize \phantom{0}&\scriptsize \phantom{0}&\scriptsize 1&\scriptsize 1&\scriptsize 1&\scriptsize 1&\scriptsize 1&\scriptsize \phantom{0}&\scriptsize 1&\scriptsize 1&\scriptsize 1&\scriptsize 1 \\
    &0\;.&0&0&1&1&1&1&1&0&1&1&0&1 & $= M_A$ \\
$+$ &0\;.&0&0&1&1&1&0&1&0&1&1&1&1 & $= M_B$ \\ \hline
    &0\;.&0&1&1&1&1&0&0&1&1&1&0&0 & $= M_C$
\end{tabular}

Результат нормализован.

\begin{tabular}{ccc} \hline \multicolumn{1}{|c}{{0}} & \multicolumn{1}{|c|}{1\:0\:0\:0\:0\:1\:0} & \multicolumn{1}{|c|}{0\:1\:1\:1\:1\:0\:0\:1\:1\:1\:0\:0} \\ \hline \scriptsize 0 & \scriptsize 1 \hfill \scriptsize 7 & \scriptsize 8 \hfill \scriptsize 19 \end{tabular}
\begin{flalign*}
    C^* &= S_M * M_M * 16^{P_M} = 121.75 \\
    C_T &= A + B = 121.7600000000000051159076975 \\
    \Delta C &= C_T - C^* = 0.0100000000000051159076975 \\
    \delta C &= \left| \frac{\Delta C}{C_T}\right| * 100\% \approx 0.008213\% &&
\end{flalign*}
\end{minipage}\hspace{0.5cm}
\begin{minipage}[t]{0.5\textwidth - 0.25cm}
\setlength{\parskip}{0.5em}
\subsection*{Формат Ф2 (с 11 битной мантиссой)}
$A = 62.84 = 0.\text{1111101101011}..._{2}$ \\ $A \approx 0.\underbracket{\text{111110110110}}_{\mathclap{\text{мантисса}}}\vphantom{0}_{\scriptstyle 2} * 2^{6}$ \\ $X_A = 6 + 128 = \underbracket{134}_{\mathclap{\text{характеристика}}}$

\begin{tabular}{ccc} \hline \multicolumn{1}{|c}{{0}} & \multicolumn{1}{|c|}{1\:0\:0\:0\:0\:1\:1\:0} & \multicolumn{1}{|c|}{1\:1\:1\:1\:0\:1\:1\:0\:1\:1\:0} \\ \hline \scriptsize 0 & \scriptsize 1 \hfill \scriptsize 8 & \scriptsize 9 \hfill \scriptsize 19 \end{tabular}

$B = 58.92 = 0.\text{1110101110101}..._{2}$ \\ $A \approx 0.\underbracket{\text{111010111011}}_{\mathclap{\text{мантисса}}}\vphantom{0}_{\scriptstyle 2} * 2^{6}$ \\ $X_B = 6 + 128 = \underbracket{134}_{\mathclap{\text{характеристика}}}$

\begin{tabular}{ccc} \hline \multicolumn{1}{|c}{{0}} & \multicolumn{1}{|c|}{1\:0\:0\:0\:0\:1\:1\:0} & \multicolumn{1}{|c|}{1\:1\:0\:1\:0\:1\:1\:1\:0\:1\:1} \\ \hline \scriptsize 0 & \scriptsize 1 \hfill \scriptsize 8 & \scriptsize 9 \hfill \scriptsize 19 \end{tabular}

\textbf{Сравнение порядков:} \\
\setlength{\tabcolsep}{2pt}
\begin{tabular}{rccccccccl}
    \scriptsize \phantom{0}&\scriptsize \phantom{0}&\scriptsize \phantom{0}&\scriptsize \phantom{0}&\scriptsize \phantom{0}&\scriptsize \phantom{0}&\scriptsize \phantom{0}&\scriptsize \phantom{0} \\
    &1&0&0&0&0&1&1&0 & $= X_A$ \\
$-$ &1&0&0&0&0&1&1&0 & $= X_B$ \\ \hline
    &0&0&0&0&0&0&0&0 & $= (X_A - X_B)_\text{доп}$
\end{tabular}

$X_A - X_B = 0,\: X_C = X_A = 134$


\textbf{Оба операнда положительные:}


\setlength{\tabcolsep}{2pt}
\begin{tabular}{rcccccccccccccl}
    \scriptsize \phantom{0}&\scriptsize 1&\scriptsize 1&\scriptsize 1&\scriptsize 1&\scriptsize 1&\scriptsize \phantom{0}&\scriptsize 1&\scriptsize 1&\scriptsize 1&\scriptsize 1&\scriptsize 1&\scriptsize \phantom{0} \\
    &0\;.&1&1&1&1&1&0&1&1&0&1&1&0 & $= M_A$ \\
$+$ &0\;.&1&1&1&0&1&0&1&1&1&0&1&1 & $= M_B$ \\ \hline
    &1\;.&1&1&1&0&0&1&1&1&0&0&0&1 & $= M_C$
\end{tabular}

Результат денормализован влево.
\begin{flalign*}
    M_C &= 1\:.\:1\:1\:1\:0\:0\:1\:1\:1\:0\:0\:0\:1\\
    M_C \rightarrow 1 &= 0\:.\:1\:1\:1\:1\:0\:0\:1\:1\:1\:0\:0\:0&&
\end{flalign*}
Характеристику результата нужно увеличить на 1: $X_C' = X_C + 1 = 135$

\begin{tabular}{ccc} \hline \multicolumn{1}{|c}{{0}} & \multicolumn{1}{|c|}{1\:0\:0\:0\:0\:1\:1\:1} & \multicolumn{1}{|c|}{1\:1\:1\:0\:0\:1\:1\:1\:0\:0\:0} \\ \hline \scriptsize 0 & \scriptsize 1 \hfill \scriptsize 8 & \scriptsize 9 \hfill \scriptsize 19 \end{tabular}
\begin{flalign*}
    C^* &= S_M * M_M * 2^{P_M} = 121.75 \\
    C_T &= A + B = 121.7600000000000051159076975 \\
    \Delta C &= C_T - C^* = 0.0100000000000051159076975 \\
    \delta C &= \left| \frac{\Delta C}{C_T}\right| * 100\% \approx 0.008213\% &&
\end{flalign*}
\end{minipage}
\newpage
\begin{minipage}[t]{0.5\textwidth - 0.25cm - 3pt}
\setlength{\parskip}{0.5em}

\textbf{A положительно, B отрицательно}


\setlength{\tabcolsep}{2pt}
\begin{tabular}{rcccccccccccccl}
    \scriptsize \phantom{0}&\scriptsize \phantom{0}&\scriptsize \phantom{0}&\scriptsize \phantom{0}&\scriptsize \phantom{0}&\scriptsize \phantom{0}&\scriptsize \phantom{0}&\scriptsize 1&\scriptsize 1&\scriptsize 1&\scriptsize 1&\scriptsize 1&\scriptsize \phantom{0} \\
    &0\;.&0&0&1&1&1&1&1&0&1&1&0&1 & $= M_A$ \\
$-$ &0\;.&0&0&1&1&1&0&1&0&1&1&1&1 & $= M_B$ \\ \hline
    &0\;.&0&0&0&0&0&0&1&1&1&1&1&0 & $= M_C$
\end{tabular}

Результат денормализован вправо.
\begin{flalign*}
    M_C &= 0\:.\:0\:0\:0\:0\:0\:0\:1\:1\:1\:1\:1\:0\\
    M_C \leftarrow 4 &= 0\:.\:0\:0\:1\:1\:1\:1\:1\:0\:0\:0\:0\:0&&
\end{flalign*}
Характеристику результата нужно уменьшить на 1: $X_C' = X_C - 1 = 65$

\begin{tabular}{ccc} \hline \multicolumn{1}{|c}{{0}} & \multicolumn{1}{|c|}{1\:0\:0\:0\:0\:0\:1} & \multicolumn{1}{|c|}{0\:0\:1\:1\:1\:1\:1\:0\:0\:0\:0\:0} \\ \hline \scriptsize 0 & \scriptsize 1 \hfill \scriptsize 7 & \scriptsize 8 \hfill \scriptsize 19 \end{tabular}
\begin{flalign*}
    C^* &= S_M * M_M * 16^{P_M} = 3.875 \\
    C_T &= A + B = 3.920000000000001705302565824 \\
    \Delta C &= C_T - C^* = 0.045000000000001705302565824 \\
    \delta C &= \left| \frac{\Delta C}{C_T}\right| * 100\% \approx 1.147959\% &&
\end{flalign*}

\textbf{A отрицательно, B положительно}


\setlength{\tabcolsep}{2pt}
\begin{tabular}{rcccccccccccccl}
    \scriptsize \phantom{0}&\scriptsize \phantom{0}&\scriptsize \phantom{0}&\scriptsize \phantom{0}&\scriptsize \phantom{0}&\scriptsize \phantom{0}&\scriptsize \phantom{0}&\scriptsize 1&\scriptsize 1&\scriptsize 1&\scriptsize 1&\scriptsize 1&\scriptsize \phantom{0} \\
    &0\;.&0&0&1&1&1&1&1&0&1&1&0&1 & $= M_A$ \\
$-$ &0\;.&0&0&1&1&1&0&1&0&1&1&1&1 & $= M_B$ \\ \hline
    &0\;.&0&0&0&0&0&0&1&1&1&1&1&0 & $= M_C$
\end{tabular}

Результат денормализован вправо.
\begin{flalign*}
    M_C &= 0\:.\:0\:0\:0\:0\:0\:0\:1\:1\:1\:1\:1\:0\\
    M_C \leftarrow 4 &= 0\:.\:0\:0\:1\:1\:1\:1\:1\:0\:0\:0\:0\:0&&
\end{flalign*}
Характеристику результата нужно уменьшить на 1: $X_C' = X_C - 1 = 65$

\begin{tabular}{ccc} \hline \multicolumn{1}{|c}{{1}} & \multicolumn{1}{|c|}{1\:0\:0\:0\:0\:0\:1} & \multicolumn{1}{|c|}{0\:0\:1\:1\:1\:1\:1\:0\:0\:0\:0\:0} \\ \hline \scriptsize 0 & \scriptsize 1 \hfill \scriptsize 7 & \scriptsize 8 \hfill \scriptsize 19 \end{tabular}
\begin{flalign*}
    C^* &= S_M * M_M * 16^{P_M} = -3.875 \\
    C_T &= A + B = -3.920000000000001705302565824 \\
    \Delta C &= C_T - C^* = -0.045000000000001705302565824 \\
    \delta C &= \left| \frac{\Delta C}{C_T}\right| * 100\% \approx 1.147959\% &&
\end{flalign*}
\end{minipage}
\begin{minipage}[t]{0.5\textwidth - 0.25cm}
\setlength{\parskip}{0.5em}

\textbf{A положительно, B отрицательно}


\setlength{\tabcolsep}{2pt}
\begin{tabular}{rcccccccccccccl}
    \scriptsize \phantom{0}&\scriptsize \phantom{0}&\scriptsize \phantom{0}&\scriptsize \phantom{0}&\scriptsize \phantom{0}&\scriptsize 1&\scriptsize 1&\scriptsize 1&\scriptsize 1&\scriptsize 1&\scriptsize \phantom{0}&\scriptsize 1&\scriptsize 1 \\
    &0\;.&1&1&1&1&1&0&1&1&0&1&1&0 & $= M_A$ \\
$-$ &0\;.&1&1&1&0&1&0&1&1&1&0&1&1 & $= M_B$ \\ \hline
    &0\;.&0&0&0&0&1&1&1&1&1&0&1&1 & $= M_C$
\end{tabular}

Результат денормализован вправо.
\begin{flalign*}
    M_C &= 0\:.\:0\:0\:0\:0\:1\:1\:1\:1\:1\:0\:1\:1\\
    M_C \leftarrow 4 &= 0\:.\:1\:1\:1\:1\:1\:0\:1\:1\:0\:0\:0\:0&&
\end{flalign*}
Характеристику результата нужно уменьшить на 4: $X_C' = X_C - 4 = 130$

\begin{tabular}{ccc} \hline \multicolumn{1}{|c}{{0}} & \multicolumn{1}{|c|}{1\:0\:0\:0\:0\:0\:1\:0} & \multicolumn{1}{|c|}{1\:1\:1\:1\:0\:1\:1\:0\:0\:0\:0} \\ \hline \scriptsize 0 & \scriptsize 1 \hfill \scriptsize 8 & \scriptsize 9 \hfill \scriptsize 19 \end{tabular}
\begin{flalign*}
    C^* &= S_M * M_M * 2^{P_M} = 3.921875 \\
    C_T &= A + B = 3.920000000000001705302565824 \\
    \Delta C &= C_T - C^* = -0.001874999999998294697434176 \\
    \delta C &= \left| \frac{\Delta C}{C_T}\right| * 100\% \approx 0.047832\% &&
\end{flalign*}

\textbf{A отрицательно, B положительно}


\setlength{\tabcolsep}{2pt}
\begin{tabular}{rcccccccccccccl}
    \scriptsize \phantom{0}&\scriptsize \phantom{0}&\scriptsize \phantom{0}&\scriptsize \phantom{0}&\scriptsize \phantom{0}&\scriptsize 1&\scriptsize 1&\scriptsize 1&\scriptsize 1&\scriptsize 1&\scriptsize \phantom{0}&\scriptsize 1&\scriptsize 1 \\
    &0\;.&1&1&1&1&1&0&1&1&0&1&1&0 & $= M_A$ \\
$-$ &0\;.&1&1&1&0&1&0&1&1&1&0&1&1 & $= M_B$ \\ \hline
    &0\;.&0&0&0&0&1&1&1&1&1&0&1&1 & $= M_C$
\end{tabular}

Результат денормализован вправо.
\begin{flalign*}
    M_C &= 0\:.\:0\:0\:0\:0\:1\:1\:1\:1\:1\:0\:1\:1\\
    M_C \leftarrow 4 &= 0\:.\:1\:1\:1\:1\:1\:0\:1\:1\:0\:0\:0\:0&&
\end{flalign*}
Характеристику результата нужно уменьшить на 4: $X_C' = X_C - 4 = 130$

\begin{tabular}{ccc} \hline \multicolumn{1}{|c}{{1}} & \multicolumn{1}{|c|}{1\:0\:0\:0\:0\:0\:1\:0} & \multicolumn{1}{|c|}{1\:1\:1\:1\:0\:1\:1\:0\:0\:0\:0} \\ \hline \scriptsize 0 & \scriptsize 1 \hfill \scriptsize 8 & \scriptsize 9 \hfill \scriptsize 19 \end{tabular}
\begin{flalign*}
    C^* &= S_M * M_M * 2^{P_M} = -3.921875 \\
    C_T &= A + B = -3.920000000000001705302565824 \\
    \Delta C &= C_T - C^* = 0.001874999999998294697434176 \\
    \delta C &= \left| \frac{\Delta C}{C_T}\right| * 100\% \approx 0.047832\% &&
\end{flalign*}
\end{minipage} \vspace{1cm}

Причины возникновения погрешности:
\begin{enumerate}
    \item Неточное представление операндов.
    \item Потеря значащих разрядов мантиссы одного из операндов при уравнивании порядков.
    \item Потеря значащих разрядов мантиссы результата при его нормализации сдвигом мантиссы вправо.
\end{enumerate}
В формате Ф2 результаты точнее, потому что операнды представлены точнее, и сдвиг при нормализации результата производится на любое число бит, не обязательно кратное 4, за счет чего теряется меньше значащих разрядов.
\end{document}
